Para Mayo

*Introducción
-¿De que es? y ¿en que consiste?
Desde hace por lo menos 100.000 años, el ser humano, se ha buscado la necesidad de crear un mercado para poder intercambiar bienes y servicios. Al principio, la forma de comerciar era a través del trueque, el intercambio directo de objetos o servicios por otros. El problema era que este tipo de negociación podía resultar ardua, costosa y lenta, porque no siempre ambas partes tenían el interés en este intercambio directo, implicando a terceros para conseguir que la negociación se haga efectiva. Por ejemplo, un agricultor podía tener interés en intercambiar su cosecha por cabezas de ganado, pero al ganadero le podía interesar más tener madera y no el cultivo que le ofrecen, entonces el agricultor debía de buscar alguna persona con madera interesada en negociar con él y a si poder finalizar el trato con el ganadero.
 
Existen algunos bienes que son más fáciles de intercambiar que otros. Estos no se reclaman por su utilidad sino por su liquidez, creándose un mercado en el que la moneda de cambio no es propio dinero. Un claro ejemplo, es el intercambio de cigarrillos que circula en las prisiones, que incluso sirve a reclusos que no fuman. Este tipo de mercados surgen por la escasez de un bien que se convierte en la ``moneda oficial''. Esta necesidad generalizada que une a toda la comunidad, posibilita el intercambio de objetos y servicios. Otro ejemplo, es el de los chocolates en Europa después de la Segunda Guerra Mundial.

Desde la aparición de las primeras monedas grabadas, allá por el siglo VII a. C. en Lidia (la actual Turquía), hasta los billetes y las monedas de hoy en día, el dinero es el medio físico para poder comerciar entre particulares.

Luego los libio apatecion el dinero las primeas monedas grabadas, cutyo valo era el oro o la plata,
Antiguamente, la forma común de comerciar era el sistema del trueque, intercambiando directamente bienes y servicios por otros. Este sistema es ineficiente y lo podemos ver con el siguiente ejemplo:
Desde hace miles de años se viene compartiendo el dinero, o cooperando entre tal y tal, la estrategia mas Tic-for Tac, es absurdo pensar que la existencia de algo centraliazdo es bueno, es absurdo pensar que la existen cia de intermdiarios es buena. Habla de la existencia de la moneda y de los intercambios de pares (musica, videos, trueque). La entidad financiera es un invento mas o menos moderno. Es un concepto tan antiguao poder interactura con tus iguales. Hablar de los microcreditos.
Hablar de microcreditos, premio nobel.
Shoali es una aplicación web que permite gestionar, de manera descentralizada, el préstamo de dinero entre particulares. Explicar el P2Plending, explicar la cooperacion, explicar que es una tecnica antigua de hace tiempo que no es nada nuevo. Red de redes
El objetivo es que cada persona sea dueño de su propio dinero, y pueda dejarlo de forma transparente como se quiera. La aplicación debe de ser capaz de permitir a cada usuario, realizar préstamos de manera sencilla, a cualquier persona o grupo de personas.
-¿motivacion? ¿porque lo hago?
Porque es genial para obtener otro metodo de financiacion no monopolizado por los intermediarios, para quie la gente pueda utilizar el P2P para intercambiar un bien como la moneda. Por que los sistemas que existen hasta ahora no son Free. Porque creo que la cooperacion y los micro creditos son un metodo muy util de evolucion humana.
Futuras investigaciones
Un nuevo tipo de mercado
-tecnologias
Python
Django
Postgres o una base de datos.
BitCoin
Distribuido, federativas (red de redes)
-estructura, más adelante
*Objetivos
Principal como que se lo contases a tu abuela
Subobjetivos (6 a 12) un parrafo cada uno, pueden ser web site, tecnologicas y no tecnologicas ¿como hacer esto? (karl fogel) investigarlo, p.e (Street performance protocol)
Planificacion
*Diseño e implementación (hasta marzo) Blog ir poniendo lo que se va haciendo.
*Resultados y pruebas (desde marzo)
*Conclusiones y trabajos futuros, espejo del capitulo 2, los problemas de la planificacion, soluciones y otros problemas).
