Para Mayo

*Introducción
1. ¿Qué? (what)
Es una plataforma que permite a cualquier persona ser dueño de su propio dinero
Es un software diseñado para que los usuarios puedan prestar su dinero….
2. ¿Quién? (who) 
Lo hace. 
Lo usa.
El que presta
El que pide.
El que gestiona.
-¿De que es? y ¿en que consiste?
Desde hace por lo menos 100.000 años, el ser humano, se ha buscado la necesidad de crear un mercado para poder intercambiar bienes y servicios. Al principio, la forma de comerciar era a través del trueque, un intercambio directo de objetos o servicios por otros. El problema era que este tipo de negociación podía resultar ardua, costosa y lenta, porque no siempre ambas partes tenían el interés en este intercambio directo, implicando a terceros para conseguir que la negociación se haga efectiva. Por ejemplo, un agricultor podía tener interés en intercambiar su cosecha por cabezas de ganado, pero al ganadero le podía interesar más tener madera y no el cultivo que le ofrecen, entonces el agricultor debía de buscar alguna persona con madera interesada en negociar con él y a si poder finalizar el trato con el ganadero. 

Con el tiempo, sucedió, que ciertos objetos eran más fáciles de intercambiar que otros, no por su utilidad precisamente, sino por su ``liquidez'' o capacidad de transformación en otro bien.  Aparecieron mercados cuyas monedas de cambio eran objetos como comida, conchas, ropa, metales, plumas, piedras preciosas, etcétera. Hoy en día, sigue existiendo este tipo de negociación, en el que un objeto suple al papel moneda. Un claro ejemplo, es el intercambio de cigarrillos que circula en las prisiones, que incluso sirve a reclusos que no fuman. Este tipo de mercados surgen por la escasez de un bien que se convierte en la ``moneda oficial''. Esta necesidad generalizada que une a toda la comunidad, posibilita el intercambio de objetos y servicios. Otro ejemplo, es el de los chocolates en Europa después de la Segunda Guerra Mundial.

El oro y la plata eran atractivos, ligeros, y fáciles de transportar, con lo que rápidamente se convirtieron en objeto de cambio. Por su maleabilidad, aparecen las primeras monedas grabadas, allá por el siglo VII a. C. en Lidia (la actual Turquía), y con ello los primeros mercados con lo que hoy conocemos por ``dinero''. La moneda era un patrón certificado de oro o plata con un peso y una pureza. 

Actualmente siguen existiendo las monedas, aunque ya no son de oro y plata. Las monedas de oro y plata desaparecen en el momento que aparece el papel moneda, documento que legitimaba una equivalencia en oro. Con la entrada de los billetes en los mercados, ya no hace falta ni trasladar las pesadas monedas de oro, ni trabajarlas para su uso legal. Los primeros ``bancos'' o guardianes de oro, son entidades que podían emitir billetes a sus clientes, a cambio de la custodia de su oro.

Al principio el modelo de negocio de un banco, era transparente. Ellos guardan tú ``oro'' o dinero, a cambio de una renta por este servicio. Pero esto se transformó, en el momento en el que se permiten la licencia de operar con tú dinero, haciendo prestamos y generando deuda. Tú dinero, ya no está controlado por ti.
 
Con la aparición de Internet, los mercados se hacen mucho más grandes. Ahora puedes comerciar, con casi cualquier persona del mundo que tenga acceso a Internet. Hasta la aparición de una moneda virtual Bitcoin \ref{bitcoin_2009}, el 24 de mayo de 2009, la gente no podía hacer compras descentralizadas por Internet, siempre debía pasar por una entidad financiera que gestiona la transacción, impidiendo realizar una transacción como en el mundo físico en la cual tu das dinero a otro en la misma mano, sin pasar a través de nadie. esto permite usar el mercado de interenet sin ser controlado por nadie. Concepto de anonimicidad muy importante ponerlo.

CoinWorked pagina que extrae bitcoin de tu pagina web.????

Moneda alternativas, monedas sociales, ECO, PUMA, FIORITO, RUPI, EPI, y PEPA entre otras. Time dollar o banco de timepo y trade dollar es la moneda de plata

Mirar ¿que es el virtapay?, ¿¿¿¿ripple es un sistema parecido al mio, es genial es una red de confianza para dar crédito, mirarlo bien??

Miarar anarcocapitalismo, criptoanarquia, cuestiones de descentralizar las cosas

Estudio de la comparativa de las redes aleatorias (solo microcreditos) , de las redes de vecinos mas cercanos (ripple) y los small worlds (mio)

Estudio de la abaricia de los individuos, si ponen mas intereses que los bancos.

Se puede estar negociando con tú dinero en cosas que van en contra de tus principios o que desconozcas. actualmente no sucede esto ya que el banco se permite la licencia de negociar con todo el dinero que entra en sus arcas.

PANICO DE LOS AHORRADORES

Habria que hablar de los banco sy de sus robos y todas sub movidas, como lo de que negocian con el dinero de la peña, sin tan si quiera tenerlo. O lo de la emision del dinero 1/9 y lo de los billetes y la cantidad de oro de las arca<s, que se dupllica la cantidad.

Desde los primeros bancos, hasta ahora, el mercado funciona con billetes y monedas o ``dinero''. La gente negocia casi siempre con dinero, aunqeue ultimamente estan a paereciendo otros tipod de mercados como son el de tiempo y el de blabla, ...


Despues de esta breve reseña historica de como ha evoluciona do el dinero y los mercados, expongo porque el interes de mi proyecto. Teniendo en cuenta que hemos tenido la necesidad de negociar desde hace mucho tiempo, pienso que actualmente en la era digital y con las nuevas tecnolpgias qye estran saliendo veo la necesidad de poder negociar con otros a traves de internte, sin necesidad de intermediarios que sea tal cual es eun mercado del mundo fisico en el cual yo ofrezco un buillete a otro y el me da el bien al coste que hemos negociado. Actualmente dsiponemos de una moneda virtual que nos podría ayudar a a hacer esto, es el Bitcoin, nos permite hacer trasacciones anonimas sin que nadie se interponga ni sepa quien la hace, similar a lo que sucederia en el mundo real. la plaicacion que quiero proporcionar va mas haya, es la posibilidad de poder dar o invertir tu dinero de manera directa sin que existan intermediarios qeu especulen con tus deudas. Que seas tu el unico que gestione tu dinero. Este sistema permitiaria a pretatarios y prestamiestas hacer operaciones directas sin intermediarios, a parte este sistema se beneficiaria del sistema de microcreditos ya que la plaiocacion lo que te permitieria es que unaos cuantos pudieran prestar a uno.


 con el siguiente ejemplo:
Desde hace miles de años se viene compartiendo el dinero, o cooperando entre tal y tal, la estrategia mas Tic-for Tac, es absurdo pensar que la existencia de algo centraliazdo es bueno, es absurdo pensar que la existen cia de intermdiarios es buena. Habla de la existencia de la moneda y de los intercambios de pares (musica, videos, trueque). La entidad financiera es un invento mas o menos moderno. Es un concepto tan antiguao poder interactura con tus iguales. Hablar de los microcreditos.
Hablar de microcreditos, premio nobel.
Shoali es una aplicación web que permite gestionar, de manera descentralizada, el préstamo de dinero entre particulares. Explicar el P2Plending, explicar la cooperacion, explicar que es una tecnica antigua de hace tiempo que no es nada nuevo. Red de redes
El objetivo es que cada persona sea dueño de su propio dinero, y pueda dejarlo de forma transparente como se quiera. La aplicación debe de ser capaz de permitir a cada usuario, realizar préstamos de manera sencilla, a cualquier persona o grupo de personas.
-¿motivacion? ¿porque lo hago?
Porque es genial para obtener otro metodo de financiacion no monopolizado por los intermediarios, para quie la gente pueda utilizar el P2P para intercambiar un bien como la moneda. Por que los sistemas que existen hasta ahora no son Free. Porque creo que la cooperacion y los micro creditos son un metodo muy util de evolucion humana.
Futuras investigaciones
Un nuevo tipo de mercado
-tecnologias
Python
Django
Postgres o una base de datos.
BitCoin
Distribuido, federativas (red de redes)
-estructura, más adelante
log transparencia para hacer distintos estudios
*Objetivos
Principal como que se lo contases a tu abuela
Subobjetivos (6 a 12) un parrafo cada uno, pueden ser web site, tecnologicas y no tecnologicas ¿como hacer esto? (karl fogel) investigarlo, p.e (Street performance protocol)
Planificacion
*Diseño e implementación (hasta marzo) Blog ir poniendo lo que se va haciendo.
*Resultados y pruebas (desde marzo)
*Conclusiones y trabajos futuros, espejo del capitulo 2, los problemas de la planificacion, soluciones y otros problemas).
