\documentclass[10pt]{article}

% Puedes entregar la plantilla en inglés, si así lo deseas.
\usepackage[spanish]{babel}
\usepackage[utf8]{inputenx}

\begin{document}

\title{}
\author{Diego Pardilla}
\date{\today}


\maketitle

\begin{abstract}
Desarrollo de una nueva aplicación capaz de permitir gestionar, de manera descentralizada, el préstamo de dinero entre particulares que se distribuyen en una red. El objetivo es que cada persona sea dueño de su propio dinero, y pueda dejarlo de forma transparente como se quiera. La aplicación debe de ser capaz de permitir a cada usuario, realizar préstamos de manera sencilla, a cualquier persona o grupo de personas.   
\end{abstract}

\section{Objetivos}

El objetivo principal de este proyecto es:

\begin{center}
\bf{Crear una aplicación que permita a los usuarios hacer préstamos, sin necesidad de intermediarios.}
\end{center}

Este objetivo se puede dividir en los siguientes subobjetivos:

\begin{enumerate}
  \item Aplicación capaz de gestionar los préstamos de comunidades de gente, tanto si son de uno a uno,  de uno a varios, de varios a uno, o simplemente dejar el dinero a otros de manera altruista.\\ 
Para ello deben de existir dos tipos de perfiles:
\begin{enumerate}
    \item \textbf{Prestamista (\textit{``ant o lender''})}: Persona que suministra dinero a los demás. Este tipo de usuario, puede fiar a quien quiera y al tipo de interés que el desee.
    \item \textbf{Prestatario (\textit{``cicada o borrower''})}: Persona que desea una cantidad de dinero. El rol de este usuario, le permite decidir cuanto dinero quiere percibir y a que tipo de interés.
\end{enumerate}
  \item Un motor de búsqueda que ayude a encontrar el préstamo que mejor se adapte a los intereses de cada usuario.
  \item Gestión financiera del capital gracias a la moneda virtual Bitcoin.
\end{enumerate}


\section{Motivación}

Actualmente si deseamos realizar un préstamo o se quiere obtener financiación, de manera virtual, es imposible hacerlo sin pasar antes a través de alguna entidad financiera. Por este motivo, creo en la necesidad de poder realizar este tipo de operaciones virtuales. Ahora, no existe una plataforma digital que supla a cooperativas, que de microcréditos, y que permita financiación entre particulares. Esta aplicación pretende ser una herramienta para optimizar una manera diferente de financiación, abogando por las inversiones entre pares, atajando a las grandes entidades financieras.

\section{Tecnologías involucradas}

Las tecnologías involucradas en este proyecto son:

\begin{itemize}
  \item Lenguaje de programación Python (django), nivel básico.
  \item La gestión de las comunidades me gustaría que fuera federativa. Ningún conocimiento.
  \item Base de datos, por orden de preferencia, PostgreSQL, MariaDB, MySQL y otras.
  \item La gestión del capital, a través de Bitcoin, nivel de conocimiento básico.
\end{itemize}

\section{Planificación tentativa}

No tengo ni idea, el comienzo de la tesis sería cuanto antes.

\begin{enumerate}
  \item Paso: Creación del portal web.
  \item Paso: Creación de la aplicación web.
  \item Paso: Creación de la forja para mantener a la comunidad.
  \item Paso: Creación de los motores de búsqueda.
  \item Paso: Creación de aplicación de gestión financiera, Bitcoin. 
  \item Paso: Creación de la aplicación cliente.
  \item Paso: Testeo de la aplicación.
  \item Entrega memoria: no tengo conocimiento.
\end{enumerate}

\section{Otros}

Considero la necesidad de una reunión para la gestación y maduración del proyecto.  
%Espacio para indicar otras cuestiones que consideres de interés, como por ejemplo, si te gustaría tener otro tutor, necesitas material especial, etc.

\end{document}
