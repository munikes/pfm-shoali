\documentclass[10pt]{article}

% Puedes entregar la plantilla en inglés, si así lo deseas.
\usepackage[spanish]{babel}
\usepackage[utf8]{inputenx}

\begin{document}

\title{}
\author{Diego Pardilla}
\date{\today}


\maketitle

\begin{abstract}
Aplicación que gestiona la deuda contraída con una entidad financiera y es capaz de calcular la mejor manera de subsanarla en función de la comunidad de deudores generada. Con esta aplicación, lo que se pretende es, minimizar el impacto de los intereses de cualquier préstamo contraído con cualquier entidad financiera. Esto se consigue creando comunidades capaces de gestionar las deudas de manera conjunta. En estas comunidades puede haber distintos perfiles: deudor, colaborador y cooperador. Deudor, persona que comparte la deuda contraída. Colaborador, persona que aporta su granito de arena sin ningún bien a cambio. Y finalmente el cooperador, persona que se hace participe de la deuda sin tener ninguna, pero anteponiéndose a futuras necesidades.
\end{abstract}

\section{Objetivos}

El objetivo principal de este proyecto es:

\begin{center}
\bf{Crear una aplicación que auto-gestione el capital de las distintas comunidades, sin necesidad de la intervención de entidades financieras}
\end{center}

Este objetivo se puede dividir en los siguientes subobjetivos:

\begin{enumerate}
  \item Subobjetivo 1. Creación de una forja para la aplicación planteada.
  \item Subobjetivo 2. Desarrollo de una aplicación en comunidad (si existe), bajo una licencia FLOSS.
  \item Subobjetivo 3. Testeo de la aplicación, gracias a beta-testers voluntarios (si existen).
  \item Subobjetivo 4. Aportación de una nueva aplicación FLOSS que pueda ser usada en beneficio de la libertad.
\end{enumerate}


\section{Motivación}

Actualmente, la palabra que más se oye, es la palabra CRISIS. No se porque deberíamos de alarmarnos, ya que sí somos honrados y aceptamos las reglas del juego, no deberíamos de preocuparnos de nada, o eso es lo que nos han prometido siempre. Yo como no creo en esta pantomima, y pienso que las reglas del juego están echas para unos pocos, acepto las reglas pero no las comparto. Por eso creo que se debe de cambiar su tratamiento. No fijarnos tanto en la competencia sino mucho más en la cooperación. Si nos fijamos en modelos de teoría de juegos (ciervo y conejos) o en modelos matemáticos de \textit{"swarm-inteligence"} o en el comportamiento de la supervivencia de alguna especie, vemos que muchas veces la competencia no es la mejor solución. Un ejemplos real que nos compete, es el desarrollo de aplicaciones bajo licencia de software libre. Me gustaría que los usuarios de mi aplicación utilizaran las redes sociales para minimizar el impacto de sus deudas.

Después de este brote megalómano. Simplemente desearía, como objetivo mas cercano y real, que mi grupo de amigos podamos ser capaces de gestionar nuestras hipotecas y que los intereses de estas, nos hagan el menor daño posible.

\section{Tecnologías involucradas}

Las tecnologías involucradas en este proyecto son:

\begin{itemize}
  \item Tecnología 1. Lenguaje de programación Python, nivel básico.
  \item Tecnología 2. Módulo de Diaspora, ningún conocimiento.
  \item Tecnología 3. Base de datos, por orden de preferencia, PostgreSQL, MariaDB, MySQL y otras.
\end{itemize}


\section{Planificación tentativa}

No tengo ni idea, el comienzo de la tesis sería cuanto antes. Pero teniendo en cuenta la carga docente se me va a hacer muy difícil comenzar antes de la finalización de las asignaturas del Máster.

\begin{enumerate}
  \item Paso 1: Creación de la forja.
  \item Paso 2: Implementación del modulo de Diaspora.
  \item Paso 3: Creación de la aplicación cliente.
  \item Paso 4: Testeo de la aplicación.
  \item Entrega memoria: ni idea
\end{enumerate}

\section{Otros}

Considero la necesidad de una reunión para la gestación y maduración del proyecto.  
%Espacio para indicar otras cuestiones que consideres de interés, como por ejemplo, si te gustaría tener otro tutor, necesitas material especial, etc.

\end{document}
