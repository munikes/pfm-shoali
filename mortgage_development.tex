\documentclass[10pt]{article}

% Puedes entregar la plantilla en inglés, si así lo deseas.
\usepackage[spanish]{babel}
\usepackage[utf8]{inputenx}

\begin{document}

\title{}
\author{Diego Pardilla}
\date{\today}


\maketitle

\begin{abstract}
Desarrollo de una nueva aplicación que gestiona la mejor amortización posible de las hipotecas de un grupo de personas. Con esta aplicación, lo que se pretende es, minimizar el impacto de los intereses generados por un grupo de hipotecas. Esto se consigue creando comunidades capaces de gestionar las hipotecas de manera conjunta. En estas comunidades puede haber distintos perfiles: deudor, colaborador y cooperador. Deudor: persona que tiene una hipoteca. Colaborador: persona que aporta su granito de arena sin ningún bien a cambio. Y, finalmente, el cooperador: persona que se hace participe de la hipoteca de los demás y beneficiario de la deuda contraída.
\end{abstract}

\section{Objetivos}

El objetivo principal de este proyecto es:

\begin{center}
\bf{Crear una aplicación que auto-gestione un conjunto de hipotecas para minimizar el interés generado}
\end{center}

Este objetivo se puede dividir en los siguientes subobjetivos:

\begin{enumerate}
  \item Aplicación capaz de gestionar comunidades de gente, dispuesta a beneficiarse de la cooperación o que simplemente quiere colabora de manera altruista.\\ 
En esta comunidad existen tres perfiles:
\begin{enumerate}
\item \textbf{Deudor}: Persona que tiene una hipoteca, su cuota será la misma que tiene en la entidad financiera o mayor, si lo desea. Su compromiso de permanencia será el mismo que tiene con la entidad financiera.
\item \textbf{Cooperador}: Persona que desea cooperar, para beneficiarse de los intereses no sustraídos por las entidades financieras. Su cuota y compromiso puede ser variable, y en función de esta aportación tendrán más o menos beneficios.
\item \textbf{Colaborador}: Persona que desea colaborar de manera altruista. No tiene cuotas, ni ningún tipo de obligación para con la comunidad.
\end{enumerate}
  \item Un motor que calcula la mejor manera de amortizar las hipotecas.
  \item Un motor que calcula los beneficios, en función de la aportación de cada miembro.
  \item Gestión financiera del capital de cada comunidad (Bitcoin, Flattr y dinero en entidades financieras éticas).
\end{enumerate}


\section{Motivación}

Cuando compramos una casa, debemos de hacer frente a una hipoteca y a los intereses que genera esta. Si esta operación, en vez de hacerla de manera individual, la afrontáramos con nuestra gente más allegada o incluso, gracias a las redes sociales, con un grupo de personas que quisiera ayudarnos, los intereses generados por nuestra hipoteca se reducirían considerablemente, o incluso podrían llegar a ser cero. Pienso, que algunas veces, debemos de olvidarnos del individualismo y la competencia, para intentar cooperar e incluso colaborar con nuestros semejantes. Estamos pagando las consecuencias de las practicas abusivas de las entidades financieras y de un sistema que aboga por tener cada vez más y más. Actualmente no tenemos un modelo alternativo, aunque varias son las tentativas: cooperativas, microcréditos, etc. Esta aplicación pretende ser una herramienta para optimizar una manera diferente de financiación.

\section{Tecnologías involucradas}

Las tecnologías involucradas en este proyecto son:

\begin{itemize}
  \item Lenguaje de programación Python, nivel básico.
  \item La gestión de las comunidades me gustaría que fuera a través de Diaspora. Ningún conocimiento.
  \item Base de datos, por orden de preferencia, PostgreSQL, MariaDB, MySQL y otras.
  \item La gestión del capital, a través de Bitcoin y Flattr, nivel de conocimiento básico.
\end{itemize}

\section{Planificación tentativa}

No tengo ni idea, el comienzo de la tesis sería cuanto antes. Pero teniendo en cuenta la carga docente se me va a hacer muy difícil comenzar antes de la finalización de las asignaturas del Máster.

\begin{enumerate}
  \item Paso: Creación de la forja.
  \item Paso: Creación de los motores de cálculo (amortización y de beneficios).
  \item Paso: Implementación del módulo de Diaspora.
  \item Paso: Creación de aplicación de gestión financiera. 
  \item Paso: Creación de la aplicación cliente.
  \item Paso: Testeo de la aplicación.
  \item Entrega memoria: no tengo conocimiento.
\end{enumerate}

\section{Otros}

%Considero la necesidad de una reunión para la gestación y maduración del proyecto.  
%Espacio para indicar otras cuestiones que consideres de interés, como por ejemplo, si te gustaría tener otro tutor, necesitas material especial, etc.

\end{document}
