\documentclass[10pt]{article}

% Puedes entregar la plantilla en inglés, si así lo deseas.
\usepackage[spanish]{babel}
\usepackage[utf8]{inputenx}

\begin{document}

\title{}
\author{Diego Pardilla}
\date{\today}


\maketitle

\begin{abstract}
Aplicación que gestiona la mejor amortización de las hipotecas de un grupo de personas. Con esta aplicación, lo que se pretende es, minimizar el impacto de los intereses generadospor el grupo de hipotecas. Esto se consigue creando comunidades capaces de gestionar las hipotecas de manera conjunta. En estas comunidades puede haber distintos perfiles: deudor, colaborador y cooperador. Deudor, persona que tiene una hipoteca. Colaborador, persona que aporta su granito de arena sin ningún bien a cambio. Y finalmente el cooperador, persona que se hace participe de la hipoteca de los demás, anteponiéndose a una futura hipoteca que pueda contraer.
\end{abstract}

\section{Objetivos}

El objetivo principal de este proyecto es:

\begin{center}
\bf{Crear una aplicación que auto-gestione un conjunto de hipotecas para minimizar el interés generado}
\end{center}

Este objetivo se puede dividir en los siguientes subobjetivos:

\begin{enumerate}
  \item Subobjetivo Aplicación capaz de gestionar comunidades de gente, dispuesta a beneficiarse de la cooperación o que simplemente colabora por amor al arte.\\ 
En esta comunidad existen tres perfiles:
\begin{enumerate}
\item \textbf{Deudor}: Persona que tiene una hipoteca, su cuota será la misma que tiene en la entidad financiera o mayor, si lo desea. Su compromiso de permanencia será el mismo que tiene con la entidad financiera.
\item \textbf{Cooperador}: Persona que desea cooperar, para beneficiarse de los intereses no sustraídos por las entidades financieras. Su cuota y compromiso puede ser variable, y en función de esta aportación tendrán más o menos beneficios.
\item \textbf{Colaborador}: Persona que desea colaborar de manera altruista. No tiene ni cuotas, ningún tipo de obligación para con la comunidad.
\end{enumerate}
  \item Subobjetivo Un motor que calcula la mejor manera de amortizar las hipotecas.
  \item Subobjetivo Un motor que calcula los beneficios, en función de la aportación de cada miembro.
  \item Subobjetivo Gestión financiera del capital de cada comunidad (Bitcoin, Flattr y dinero en entidades financieras éticas).
\end{enumerate}


\section{Motivación}

Cuando compramos una casa, debemos de hacer frente a una hipoteca y a los intereses que genera esta. Si esta operación, en vez de hacerla de manera individual, la afrontáramos con nuestra gente más allegada o incluso, gracias a las redes sociales, con un grupo de personas que quisiera ayudarnos. Los intereses generados por nuestra hipoteca, se reducirían considerablemente o incluso podrían llegar a ser cero. Pienso, que algunas veces, debemos de olvidarnos del individualismo y la competencia, para intentar cooperar e incluso colaborar con nuestros semejantes. Que debemos de dejar de pensar tanto en el más, más y más, para pensar en el mejor.

\section{Tecnologías involucradas}

Las tecnologías involucradas en este proyecto son:

\begin{itemize}
  \item Tecnología Lenguaje de programación Python, nivel básico.
  \item Tecnología La gestión de las comunidades me gustaría que fuera a través de Diaspora. Ningún conocimiento.
  \item Tecnología Base de datos, por orden de preferencia, PostgreSQL, MariaDB, MySQL y otras.
  \item Tecnología La gestión del capital, a través de Bitcoin y Flattr, nivel de conocimiento básico.
\end{itemize}


\section{Planificación tentativa}

No tengo ni idea, el comienzo de la tesis sería cuanto antes. Pero teniendo en cuenta la carga docente se me va a hacer muy difícil comenzar antes de la finalización de las asignaturas del Máster.

\begin{enumerate}
  \item Paso: Creación de la forja.
  \item Paso: Creación de los motores de cálculo (amortización y de beneficios).
  \item Paso: Implementación del módulo de Diaspora.
  \item Paso: Creación de aplicación de gestión financiera. 
  \item Paso: Creación de la aplicación cliente.
  \item Paso: Testeo de la aplicación.
  \item Entrega memoria: no tengo conocimiento.
\end{enumerate}

\section{Otros}

%Considero la necesidad de una reunión para la gestación y maduración del proyecto.  
%Espacio para indicar otras cuestiones que consideres de interés, como por ejemplo, si te gustaría tener otro tutor, necesitas material especial, etc.

\end{document}
